\documentclass[10pt]{article}
\usepackage[utf8]{inputenc}
\usepackage{cite}
\usepackage{times}
\usepackage[a4paper, total={6in, 8in}]{geometry}
\usepackage{hyperref}

\title{Securing Containerized Environments and Kubernetes-Based Deployments}
% Replace "Proposed Title" with your topic title 
\author{
  Abdiqani Abdullahi \\
  \texttt{ababd2714@usn.no}
  \and
  Sair Mohammed Nazari \\
  \texttt{247014@student.usn.no}
}
\date{Group 6 \\ [3ex] 
% Replace XX with your group number on Canvas
Project proposal for CS4220 -- Spring 2026 \\ [1ex]
%Department of Science and Industry Systems \\ [1ex]
University of South-Eastern Norway \\  [3ex] 
\today}

\pagenumbering{gobble}

\begin{document}

\maketitle


Containers are software units that package an application together with the files, libraries and the dependencies that it needs to run, which allows the application to run consistently across different environments such as the developers laptop or testing systems.

Containerization is the technique used to create and manage containers and is performed by using platforms such as Docker. To manage and orchestrate a large number of containers across multiple machines Kubernetes is used.

\vspace{2em}

The goal of this project is to compare the security measures and hardening strategies for containerized environments and Kubernetes-based deployments. With the focus on their security posture, performance and operational complexity. The project is based on these three papers \cite{Moric2025KubernetesSecurity}, \cite{Chen2025ShadowKube}, \cite{Cesarano2025KubeFence}.



\small
\bibliographystyle{IEEEtran}
\bibliography{Ref}

\end{document}
