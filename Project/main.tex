\documentclass[a4paper, 10pt]{article}
\usepackage[utf8]{inputenc}
\usepackage[includeheadfoot,margin=1in]{geometry}
\usepackage[T1]{fontenc}
\usepackage{times}
\usepackage[english]{babel}
\usepackage{graphicx}
\usepackage[export]{adjustbox}
\usepackage{subcaption}
\usepackage{wrapfig}
\usepackage{array}
\usepackage{multirow}
\usepackage{tocloft}
\usepackage{fancyhdr}
\usepackage{hyperref}
\usepackage{nopageno}
\pagenumbering{arabic}
\pagestyle{fancy}
\fancyhf{}
\fancyhead[L]{\includegraphics[width=1.5cm, height=1cm]{USNlogo}}
\fancyhead[R]{CS4220}
\cfoot{\thepage}
\title{CS4220 -- System Administration and Security \\
Project Assignments -- Part I \\[2ex] 
Securing Containerized Environments and Kubernetes-Based Deployments \\[1ex]}  
\author{
  Abdiqani Abdullahi \\
  \texttt{ababd2714@usn.no}
  \and
  Sair Mohammed Nazari \\
  \texttt{247014@usn.no}
}

\date{University of South-Eastern Norway, Campus Kongsberg \\  [2ex] \today}
\begin{document}
\maketitle
\thispagestyle{empty}

\begin{abstract}
   You can write few lines about the importance of subject and what your paper is about. 
\end{abstract}

\section{Introduction}

With the increasing adoption of cloud architecture in modern software system, it has reshaped the way systems are designed, deployed and operated.
The technology behind this is containerization, which enables applications
to be packaged with their dependencies and executed consistently across environments.

To orchestrate containerized workload at scale, Kubernetes has emerged as the standard platform, by providing powerful automation capabilities for deployment, scaling and management of containerized applications. That said, the same feature that make Kubernetes flexible and extensive also introduce security challenges. The systems distributes architecture, extensive API framework and declarative configurations make the system prone to misconfigurations, unauthorized privilege gain and unauthorized access.

The security in Kubernetes relies heavily on correct configuration by administrators and developers and is not enabled by default. There have been real-world incidents that have demonstrated that insecure details, lax role-based access control(RABC) policies and unprotected control plane can allow attackers to compromise entire clusters. With this once the attackers gain access to the Kubernetes API or a privileged workload, they can deploy malicious containers, expand access through connected systems, or abuse cluster resources. 

Recent researches highlight the common occurrence of these security threats and the limitation of existing defense mechanism. \cite{Moric2025KubernetesSecurity} demonstrates that many Kubernetes compromise stems from the control plane misconfiguration and inadequate security hardening, reinforcing the need for organized security baseline and compliance centered configurations. \cite{Chen2025ShadowKube} shows that the traditional monitoring technique and rule-based defenses are often ineffective in Kubernetes environments, where they suggest the use of behavioral monitoring and honeypot to detect and stop attacks. \cite{Cesarano2025KubeFence} points out that the role-based access control in the Kubernetes API is to broad to protect, and recommends the use of API filtering to reduce the attacks. These studies show that there is not a single security system that will protect Kubernetes, but a combination of preventive hardening, reducing attacks surfaces and active defenses during runtime.

By building on these insight, this paper provides a systematic analysis of security hardening techniques and security strategies for containerized environments and Kubernetes based deployments.

This paper aims to:

\begin{itemize}
    \item Examine the key security vulnerabilities and potential attack surface for Kubernetes.
    \item Review and analyze recent approaches designed to address these issues.
    \item Compare these approaches, highlighting their strength, weaknesses, and aspects of deploying them.
    \item Present challenges and outline potential direction for future research to strengthen Kubernetes clusters.
\end{itemize}

\section{Other sections and subsections}
Add new sections (or subsections) as needed. You can describe challenges and open problems, solutions and future directions, etc. You can present some solutions to the security problems described in previous sections. You can find such solutions from recent literature that you have studied. You can also describe your proposed solutions if you have any. You may present future directions if you have identified any. 

\subsection{About references} 
\begin{itemize}
	\item You should have at least 3 scholar and peer-reviewed articles from well-reputed conferences and journals from last 3 years. 
	\item Use a standard reference style as you can see in the references section. Simply download Bibtex info from \url{dblp.org} and add them to the Ref.bib file in the attachment so that your references are automatically added. You might alternatively take it from Google Scholar, but it is not as precise as DBLP: In Google Scholar: Go to ``Bibliography manager'' at \url{https://scholar.google.com/scholar_settings?} and select ``Show links to import citations into Bibtex'' and Save.

\end{itemize}

\vspace{-12pt}

\bibliographystyle{ieeetr}
\bibliography{Ref}

\end{document}