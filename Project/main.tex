\documentclass[a4paper, 10pt]{article}
\usepackage[utf8]{inputenc}
\usepackage[includeheadfoot,margin=1in]{geometry}
\usepackage[T1]{fontenc}
\usepackage{times}
\usepackage[english]{babel}
\usepackage{graphicx}
\usepackage[export]{adjustbox}
\usepackage{subcaption}
\usepackage{wrapfig}
\usepackage{array}
\usepackage{multirow}
\usepackage{tocloft}
\usepackage{fancyhdr}
\usepackage{hyperref}
\usepackage{nopageno}
\pagenumbering{arabic}
\pagestyle{fancy}
\fancyhf{}
\fancyhead[L]{\includegraphics[width=1.5cm, height=1cm]{Project/Images/USNlogo.png}}
\fancyhead[R]{CS4220}
\cfoot{\thepage}
\title{CS4220 -- System Administration and Security \\
Project Assignments -- Part I \\[2ex] 
Securing Containerized Environments and Kubernetes-Based Deployments \\[1ex]}  
\author{
  Abdiqani Abdullahi \\
  \texttt{ababd2714@usn.no}
  \and
  Sair Mohammed Nazari \\
  \texttt{247014@usn.no}
}

\date{University of South-Eastern Norway, Campus Kongsberg \\  [2ex] \today}
\begin{document}
\maketitle
\thispagestyle{empty}

\begin{abstract}
   You can write few lines about the importance of subject and what your paper is about. 
\end{abstract}

%SECTION Introduction

\section{Introduction}

With the increasing adoption of cloud architecture in modern software system, it has reshaped the way systems are designed, deployed and operated.
The technology behind this is containerization, which enables applications
to be packaged with their dependencies and executed consistently across environments.

To orchestrate containerized workload at scale, Kubernetes has emerged as the standard platform, by providing powerful automation capabilities for deployment, scaling and management of containerized applications. That said, the same feature that make Kubernetes flexible and extensive also introduce security challenges. The systems distribute architecture, extensive API framework and declarative configurations make the system prone to misconfiguration, unauthorized privilege gains and unauthorized access.

The security in Kubernetes relies heavily on correct configuration by administrators and developers and is not enabled by default. There have been real-world incidents that have demonstrated that insecure details, lax role-based access control(RABC) policies and unprotected control plane can allow attackers to compromise entire clusters. With this once the attackers gain access to the Kubernetes API or a privileged workload, they can deploy malicious containers, expand access through connected systems, or abuse cluster resources. 

Recent researches highlight the common occurrence of these security threats and the limitation of existing defense mechanism. \cite{Moric2025KubernetesSecurity} demonstrates that many Kubernetes compromise stems from the control plane misconfiguration and inadequate security hardening, reinforcing the need for organized security baseline and compliance centered configurations. \cite{Chen2025ShadowKube} shows that the traditional monitoring technique and rule-based defenses are often ineffective in Kubernetes environments, where they suggest the use of behavioral monitoring and honeypot to detect and stop attacks. \cite{Cesarano2025KubeFence} points out that the role-based access control in the Kubernetes API is to broad to protect, and recommends the use of API filtering to reduce the attacks. These studies show that there is not a single security system that will protect Kubernetes, but a combination of preventive hardening, reducing attacks surfaces and active defenses during runtime.

By building on these insight, this paper provides a systematic analysis of security hardening techniques and security strategies for containerized environments and Kubernetes based deployments.

This paper aims to:

\begin{itemize}
    \item Examine the key security vulnerabilities and potential attack surface for Kubernetes.
    \item Review and analyze recent approaches designed to address these issues.
    \item Compare these approaches, highlighting their strength, weaknesses, and aspects of deploying them.
    \item Present challenges and outline potential direction for future research to strengthen Kubernetes clusters.
\end{itemize}

%SECTION 2

\section{Background and Threat Model}

This section presents the technical background on containerized systems and Kubernetes, and defines threat model used throughout the paper.

\subsection{Virtualization and containerization}

Virtualization has long been used to improve resource utilization by runnung multiple virtual machines (VM) to run on a single host. The virtual machines each included their own guest operating system, which provided strong isolation between individual workloads, but at the cost of higher resource usage consumption and slower startup times. While it did improve the infrastructure flexibility, it does not support the scalability and rapid deployment that is required by modern cloud architecture.

Containerization addressed these limitations by enabling application to execute within isolated user spaces while sharing the host operating system kernel. Container engines such as Docker use Linux built isolation and resource limiting mechanism so that containers behave like in separate environment with the same operating system kernel. Compared to the virtual machines, they are faster, more lightweight, consume less resources. However, sharing the host kernel model also introduces new security risk, such as misconfigurations and vulnerabilities that can be exploited by attackers to allow them to escape the container isolation and access the host resources. \cite{Moric2025KubernetesSecurity}

\subsection{Container orchestration and Kubernetes}

As the container application becomes more widely adopted, the need to manage a large number of containers across distributed environements led to the development of a platform to manage these containers. Kubernetes emerged as the dominant orchestration system, automating deployment, management, recovery and scaling of containerized application.

Kubernetes operates using clusters that consist of a control plane and a set of worker nodes. The control plane consist of components such as API server, scheduler, control manager and the etcd key-value store that are critical to the management of the overall state of the clusters. Worker nodes run applications workloads in the form of pods, these pods group one or more containers together and share network and storage resources. Interactions with the clusters such as deploying application or changing the configuration file is done through the Kubernetes API.

Even though Kubernetes makes application management easier and more efficient through flexibility and automation, it also expands the systems attack surface. It exposes a wide variety of functionality that, if not properly secured, can be exploited by attackers. Studies \cite{Cesarano2025KubeFence} show that many vulnerabilities and misconfigurations are caused by API features, making the Kubernetes API a target for attackers. Furthermore, because Kubernetes workloads are constantly changing, making consistent policies is difficult, as shown in real-world deployment \cite{Chen2025ShadowKube}.

\subsection{Security Model of Kubernetes}

Kubernetes several built--in security mechanism specifically for the API, such as authentication, authorization and admission control. Authentication confirms the identify of the user and services interacting with the cluster, while authorization is generally handled by role-based access control (RBAC). RBAC decides permission for Kubernetes resources such as pods, deployment, and specifies which actions the users and services can perform.

However, RBAC only gives basic control of who can access resources, but does not limit how specific settings inside those resources can be used. This makes it possible for users with legitimate access to the resources to abuse advanced configurations to gain more privileges or exploit weaknesses. Security measures, such as network policies, pod security settings, and audit logging need to be manually set up and maintained by administrators. With these controls often applied inconsistently or turned off, it contributes significantly to Kubernetes security incidents \cite{Moric2025KubernetesSecurity}.

\subsection{Threat Model}

Given the centralized role of Kubernetes in managing containerized environments, this paper focuses on a threat model that focuses on the attackers exploiting weaknesses in the Kubernetes configuration, access control mechanisms and workload isolation to gain unauthorized influence over cluster behavior. These attacks operate through compromised containers, misconfigured service accounts or exposed Kubernetes API endpoints, which allows them to issued legitimate API request with malicious intent.

The primary asset under protection in this model is the Kubernetes control plane, integrity of declarative configuration, isolation between workloads, and sensitive data such as secret or credentials. Once initial access is gained, attackers are assumed to be able to deploy and modify workloads and influence the cluster behavior through legitimate API operations. This analysis emphasizes configuration driven attacks paths and security risk that are fundamental to Kubernetes orchestration and does not consider attacks targeting the underlying infrastructure, hardware, kernel-level vulnerabilities or cloud provider services.


%SECTION 3

\section{Security Challenges in Containerized and Kubernetes Environment}

Despite it's wide adoption and powerful capabilities, Kubernetes introduces a number of security challenges which is a result of it's architectural design, configuration complexity and operational model.

\subsection{Container and Configuration Challenges}

One of the biggest security challenges with Kubernetes is the complexity of it's configuration models. Kubernetes depends on declarative configuration files, which are files where you tell the system what you want it to look like, rather than what to do. While this enables automation and scalability, it also makes mistakes more likely. To prevent security issues, administrators have to correctly configure parameters authentication, networking, storage and workload isolation.

In real-deployments, misconfigurations such as giving to many permisive RBAC roles, unsecured API endpoints, privileged containers, or setting up service accounts incorrectly are common. These misconfigurations can grant users or workloads excessive access, allowing attackers to gain control. Studies have demonstrated that many Kubernetes security challenges are not caused by unknown vulnerabilities, but by insecure default settings and configurations \cite{Moric2025KubernetesSecurity}.

\subsection{Kubernetes Control Plan and API Attack Surface}

For attackers the Kubernetes control plane is a high-value target due to it's central role in managing clusters and workload scheduling. The Kubernetes API manages all clusters and allows creating, modiyfing and deleting resources. If the API access control are not probably restricted, it may give access to attackers to abuse legitimate API calls to control workloads or manipulate cluster configuration.

Despite the fact that Kubernetes uses authentication and authorization mechanism, the API is still a critical attack surface. In some situations authenticated users with limited privileges can still misuse advanced configuration to cause it to behave unintended. A significant amount of vulnerabilities and misconfigurations are exercised through API specific features rather than the entire resources, this makes it difficult to control API access using the traditional access control methodd. Because the traditional access controls are broad it's hard to allow access and block others at a detailed level \cite{Cesarano2025KubeFence}.


\subsection{Limitations of Role-Based Access Control and Static Access Control}

The primary authorization mechanism used in Kubernetes to regulate access to cluster resources is role-based access control. The RBAC policies specify witch users or services that can perform actions on the Kubernetes resources such as pods, deployment or services. While the role-based access control provides a clear and standarized approach to access control, it has limitations in managing access in complex and rapidly changing Kubernetes environement.

A key limitation of RBAC is that it grants broad permission, without fine control over evert possible action. RBAC policies decides what kind of resources (pods, deployment or service) a user can access and what actions they can perform on them like (create, delete and modify). But it does not control the small details inside the resources such as the specific configuration option a user can choose. This means users with permission to create or modify resources can can misuse advanced configuration option to gain privileges or bypass isolation

RBAC is also a static access control, meaning it does not adjust to change in how workload behave or run. Since Kubernetes environement is dynamic, workloads are being created, scaled and removed constantly and because RBAC is a fixed, the policies cannot respond to unusual activity or misuse of granted permission at runtime. This limits it effectiveness against compromised workloads trusted users who misuse their small permissions to cause problems.

At a scale, managing these RBAC policies introduces operational challenges. To balance between security and usability when defining roles, administrators often grant people more access then they need to avoid slowing down development, which increases attack surface and weakens the clusters overall security. Recent studies \cite{Cesarano2025KubeFence} have highlighted the limitation with RBAC as a standalone defense mechanism, so additional approaches are need the provide finer control and while workloads are running.


%SECTION 4


\section{Security Hardening and Defense Strategies}

The security challenges described earlier have led to many strategies designed to improve security of containerized and Kubernetes environment. With recent studies demonstrating that relying on a single security mechanism is insufficient and requires the combination of preventive hardening, runtime and attack surface reduction. In this section we will review the security approaches focused on the control plane and workload hardening, behavioral monitoring and detailed API access control and demonstrating that they are complementary and not mutually exclusive.

\subsection{Control Plane and Workload Hardening}

A popular approach to improving Kubernetes security focuses on strengthening the control plane by applying safe default configuration. The research by \cite{Moric2025KubernetesSecurity} highlights that the primary security problems stems from misconfiguration and insecure deaults rather than unknown vulnerabilities. They emphasize on security hardening by focusing on key components like the API servers, etcd, node services and making sure Kubernetes setup complies with established security guidelines.

Control plane hardening involves measures such as securing API communication using strong authentication and encryption, limiting access to etcd, applying least-privilege RBAC policies and turning of insecure and unnecessary features. At the workload level, hardening practices involve limiting container privileges, avoiding privileged containers, enforcing non-root execution and probably configuring network policies.

This approach improves the overall Security of Kubernetes clusters by reducing the likelihood of a successful attack caused by misconfiguration. However, it relies heavily on administrators applying correct and consistent configuration and does not protect runtime attacks or misuse of legitimate permission once workloads are deployed.

\subsection{Runtime Monitoring and Behavioral Defense}

While the static hardening reduces the attack surface, it is insufficient at detecting attacks that occur during runtime, especially in highly dynamic Kubernetes environment. To address this limitation studies \cite{Chen2025ShadowKube} have proposed a behavioral monitoring approach that focuses on detecting unusual workload behavior instead of relying only on predefined rules.

Their approach, ShadowKube works by learning normal behavioral patterns of Kubernetes workloads and integrates honeypot-based deceptions mechanisms to identify malicious activity. By observing when workloads act differently from their expected behavior, the system can detect compromised containers or unauthorized actions that static security might miss. The use of honeypot would also allow the system to analyze attack behavior, improving detection accuracy.

Runtime monitoring improves security by making it possible to detect attacks that exploit legitimate access or occur after deployment. However, such approach introduces operational complexity and may cause additional performance cost. In addition, defining accurate behavioral baseline is difficult in diverse and evolving environments and poorly managed false positives can impact usability.

\subsection{API Attack Surface Reduction and Precise Access Control}

To further enhance Kubernetes security, the API attack surface must be reduced. Researcher's \cite{Cesarano2025KubeFence} argue that the Kubernetes role-based access control is to broad for adequately protecting the API, since it does not restrict how configuration fields withing the resources are used. This allows attackers to exploit advanced features of the API, even while operating withing the boundaries of assigned permission.

To address these issues, they proposed a API filtering that ensures workloads can only access the API features they need. The system looks at analyzes trusted application configurations to create safe API policies. These configurations can include Kubernetes Operators, which are programs that help manage and automate applications on the clusters. Another one is Helm Charts, which are templates that define how applications should be deployed. By analyzing these, the system can block unnecessary or risky API features while still letting normal workloads run correctly, reducing the attack surface attackers can misuse the API.

The API attack surface reduction strengthens Kubernetes security by enforcing the principle of least privilege with more detailed control then traditional RBAC. However, it depends on precise workload information and adds extra policy management that needs to be handled alongside existing RBAC rules.


\section{Comparative analysis}

The security approaches discussed in the previous sections addresses Kubernetes security challenges from different perspectives. While their goals is to improve the overall security of the Kubernetes environment, they wary in assumption, scope and operation. In this section we will compare these approaches, their effectiveness, their impact and the trade-offs they introduce in deployment and operation.

\subsection{Comparison of Security Posture}

Control plane and workload hardening, as presented in "Security Hardening and compliance assessment of Kubernetes control plane and workloads" \cite{Moric2025KubernetesSecurity}, primarily strengthens Kubernetes by reducing the risk of misconfigurations and insecure defaults. This approach prevents common configurations based attacks and lays the foundation for strong security. However, the problem is that it assumes that administrators have correctly applied and maintain hardening policies over time, since once the workload is deployed, static hardening provides limited protection against attack that exploit legitimate permissions or occur during runtime.

Runtime monitoring approaches, such as the one proposed in "Shadowkub: Enhancing Kubernetes security with behavioral monitoring and honeypot integration" \cite{Chen2025ShadowKube}, strenghten security by detecting malicious behavior that bypass static defenses. Through a continuous monitoring of workload behavior and anomaly detection, runtime defenses can uncover compromised containers or unauthorized activity even when attackers operate within granted permission. The approach is effective against runtime threats and activities following compromise, however it does not prevent misconfiguration or reduce the exposed attack surface before deployment.

API attack surface reduction, presented in "Kubefence: Security hardening of Kubernetes attack surface" \cite{Cesarano2025KubeFence}, enhances security by reducing what attackers can do from the start. By enforcing strict restrictions on API usage, this approach reduces attack paths and mitigates both misconfiguration and certain vulnerabilities. Its effectiveness depends on accurately understanding workload behavior to generate correct polices, offering stronger preventive protection than traditional RBAC.

Together, these approaches strengthen the overall security: Hardening reduces misconfiguration risk, runtime monitoring detects ongoing attacks and lastly API filtering limits exploitable functionalities.

\subsection{Operational and Security cost consideration}

From an operational view, control plane and workload hardening have little impact on performance cost, as they rely mainly on preconfigured settings. However there will be operational burden placed on the administrators can be significant, since maintaining secure configurations across multiple evolving clusters requires continuous effort and expertise. 

Building on this, runtime monitoring introduce complexity since they require continuous observation, data collection and analysis. Techniques such as behavioral monitoring and honeypot may increase system load and require fine-tuning to prevent false positives. The security benefits often outweigh the system load in this case, although for large or resource limited systems it becomes harder to manage.

In combination with runtime monitoring and workload hardening, API attack surface reduction provides a layered security approach. By limiting exposed API surface and applying least-privilege access, the potential damage from successful attacks is reduced. This meant that while runtime performance impact is low, the effectiveness of this approach depends on the accuracy of the policies as incorrect or misconfigured polices could disrupt workload or weaken the security.

\subsection{Complementary and Defense}

One important insight from this comparison is that no single approach addresses all Kubernetes security challenges. Each method addresses a distinct phase of the attack lifecycle and helps compensate for the weaknesses of the others. Control plane hardening is preventive, runtime monitoring focuses on detection and API filtering enforces restriction.

These characteristics suggest a multi-layered security that combines several defenses is better than depending on just one. Strengthening the foundational baseline reduces misconfigurations, API filtering can limit damage caused by compromised credentials and runtime monitoring can detect attacks that slips past the preventive measures. This layered approach follows established security practices and reflects the complexity of real-world Kubernetes environment.


\section{Future Research Direction}

Despite significant advances in securing containerized and Kubernetes-based environment, there are still several challenges that remain. In This section we will discuss future direction aimed at improving the security posture of Kubernetes deployment.

\subsection{Automated and Context-Aware Security Configurations}



\section{Conclusion}



\vspace{-12pt}

\bibliographystyle{ieeetr}
\bibliography{Ref}



\clearpage
\setcounter{part}{1}
\part{Case Study: Security Design for Healthcare Systems}
\setcounter{section}{0}

\section{System Description and Security Need}
In this part of the thesis, which is a case study, we have chosen to focus on the Electronic Health Record (EHR) system designed for a regional hospital. This is a digital platform for collecting and storing patient health information across different treatment sites and clinical departments. The main purpose of such a system is to provide healthcare professionals with secure and rapid access to critical patient information, including medical records, lab results, X-rays and surgical reports. This helps increase patient safety and is more efficient when it comes to interaction between doctors, nurses and other support staff.\\

The system consists of three main components, these are a client side, an application server and the database, which is built as a distributed architecture. The client side is the workstation in the hospital department, mobile devices used during visits, or protected portals for the benefit of patients. This can be either iPads or computers on trolleys. These are used by the doctor to check blood test results, show X-ray images to the patient or write new medication doses while they are with the patient. Since these devices are mobile, there is also a risk that they can be lost or stolen or that unauthorized persons see the screen. Therefore, there is a need for automatic locking and access filters. The application server is primarily responsible for handling logical operations and requests, while the database is responsible for securing the actual journal data and storing it in an encrypted format.\\

For such a system as the hospital uses, it is important to protect it from threats by implementing security mechanisms related to CIA. They consist of information that is confidentiality, data security (integrity), and ensuring that the system is always usable during critical situations, and that important information is always available to healthcare personnel (availability). Confidentiality is always a major challenge for such a system because the system contains sensitive personal information, it attracts both hackers and creates challenges among personnel when it comes to internal snooping. Integrity is also a critical challenge here, because if information about allergies or medication dosage of a patient is changed by unauthorized persons, it can lead to a direct threat to the patient's life and health. When it comes to emergency situations, accessibility is crucial, especially during such situations where medical care is about getting quick access to the patient's record. The main challenge is to implement the necessary security measures without compromising usability, because if the system becomes too cumbersome, the risk also increases that employees will use unsafe detours to complete their jobs on time.

\begin{figure}[h!]
    \centering
    % Her setter vi bredden til 100% og høyden til maks 8cm
    \includegraphics[width=0.9\linewidth, height=8cm, keepaspectratio]{image.png}
    \caption{System Architecture and Security Mechanisms}
    \vspace{10pt} % Gir litt avstand mellom bildet og forklaringen
    \small % Gjør forklaringsteksten litt mindre enn brødteksten for bedre layout
    As illustrated in Figure \ref{fig:architecture}, the security is built as a layered defense. I have chosen to implement a \textbf{Secure Zone} (dashed area) to isolate the most critical system components from the open network. The diagram also demonstrates the \textbf{negative logic} of the system; for instance, if a user provides an incorrect password or MFA code, the request is terminated at the border of the secure zone. This acts as a barrier that prevents unauthorized traffic from reaching the application logic or database, effectively reducing the risk of attacks such as SQL injections or unauthorized access.
    \label{fig:architecture}
\end{figure}

\section{Identity Management }
In a modern healthcare system, identity management is the foundation of security. It is more than just a username or password. It is a framework for protecting the right people from accessing the necessary technological resources at the right time and for the right reasons. Confidentiality is critical in an environment like a hospital, so it is crucial to have effective identity management to prevent both external attacks and internal snooping.


\subsection{Identity lifecycle}
A user's identity in the system starts long before the actual login and the process consists of three phases (provisioning, maintenance, and deprovisioning).

Provisioning: When a new doctor or nurse is hired at the regional hospital, their digital identity must be created based on verified data from the human resources (HR) system. A critical part of this process is to bind the identity to the authorization register for healthcare personnel. This means that only those who have been granted approved access also have access to the clinical modules in the EHR system.

Maintenance: This phase is about access control related to the different roles (RBAC). That is, instead of assigning rights to an individual, the rights are assigned based on roles such as doctor, nurse, or administrative employee. The role determines whether a request to the database is approved as an authorized query, which is also shown in the architecture diagram (fig.1)

Deprovisioning: When working hours are over, it is critical and necessary that access is immediately revoked. Automated deprovisioning prevents ghost accounts, i.e. inactive accounts that can be misused in a possible attack where the attacker can access sensitive data about patients undetected.

\subsection{Authentication mechanisms and MFA} As I have illustrated in the diagram (fig.1), everyone must go through an ID Provider before they can access the application server. Authentication is the process of verifying that a person is actually who they claim to be. For hospital systems, multi-factor authentication (MFA) is required to achieve a sufficient level of trust. Two out of three factors are required for multi-factor authentication (MFA).

\begin{itemize}
    \item Number one is something that only you know, it's a password or pin code.
    \item Something you always have with you or on you, such as a smart card, physical security key, or code if you use a mobile app that you have access to.
    \item The last factor is something you are yourself, such as Biometrics (fingerprints) or facial recognition
\end{itemize}

The diagram we created (fig 1) where all incorrect attempts will be caught by the negative logic. According to our diagram, the system works like this, if a doctor enters the wrong password or MFA code, the system will send the request back to the start at the border of the secure zone. This means that the person trying to gain access will not be allowed to pass through the secure zone unless they enter the correct password or MFA code. Such a security mechanism is very important and crucial when it comes to hospital systems to be able to prevent and protect against any brute force attacks where hackers try to guess credentials.

\subsection{Authentication strength and trust levels} A system like the hospital's where they handle sensitive personal data, then just logging in is not enough. This is operated at different levels of assurance. For healthcare personnel operating internally, MFA is required. This means that the identity is verified through at least two independent factors, as shown in the diagram under ID Provider. By requiring something the user knows as a password and something the user has such as a physical access card or code chip, the risk of unauthorized access if a password were to be revealed is eliminated. Patients who log in externally use national ID solutions such as BankID, which guarantee the highest level of security (level 4).


\subsection{Access review and audit (Attestation)} Identity management does not end after access is granted. To maintain high-level security over time, the system must include regular access review procedures. This means that department heads at the hospital must periodically verify that their employees still have a business need for the rights they have been granted. For example, if a nurse changes departments from surgery to psychiatry, the existing access should be cleaned up and new rights should be created from scratch. This practice is important and crucial to prevent privilege creep, where over time they accumulate excessively extensive rights. In addition, automated logs of who has accessed different identities are a legal requirement to document compliance with privacy regulations.




\section{Access Control Policy}
s


\section{Authentication Mechanism }



\section{Authorization Framework and Zero Trust}




\end{document}