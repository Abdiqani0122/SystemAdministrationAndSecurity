\documentclass[a4paper, 10pt]{article}
\usepackage[utf8]{inputenc}
\usepackage[includeheadfoot,margin=1in]{geometry}
\usepackage[T1]{fontenc}
\usepackage{times}
\usepackage[english]{babel}
\usepackage{graphicx}
\usepackage[export]{adjustbox}
\usepackage{subcaption}
\usepackage{wrapfig}
\usepackage{array}
\usepackage{multirow}
\usepackage{tocloft}
\usepackage{fancyhdr}
\usepackage{hyperref}
\usepackage{nopageno}
\pagenumbering{arabic}
\pagestyle{fancy}
\fancyhf{}
\fancyhead[L]{\includegraphics[width=1.5cm, height=1cm]{Project/Images/USNlogo.png}}
\fancyhead[R]{CS4220}
\cfoot{\thepage}
\title{CS4220 -- System Administration and Security \\
Project Assignments -- Part I \\[2ex] 
Securing Containerized Environments and Kubernetes-Based Deployments \\[1ex]}  
\author{
  Abdiqani Abdullahi \\
  \texttt{ababd2714@usn.no}
  \and
  Sair Mohammed Nazari \\
  \texttt{247014@usn.no}
}

\date{University of South-Eastern Norway, Campus Kongsberg \\  [2ex] \today}
\begin{document}
\maketitle
\thispagestyle{empty}

\begin{abstract}
   You can write few lines about the importance of subject and what your paper is about. 
\end{abstract}

\section{Introduction}

With the increasing adoption of cloud architecture in modern software system, it has reshaped the way systems are designed, deployed and operated.
The technology behind this is containerization, which enables applications
to be packaged with their dependencies and executed consistently across environments.

To orchestrate containerized workload at scale, Kubernetes has emerged as the standard platform, by providing powerful automation capabilities for deployment, scaling and management of containerized applications. That said, the same feature that make Kubernetes flexible and extensive also introduce security challenges. The systems distribute architecture, extensive API framework and declarative configurations make the system prone to misconfiguration, unauthorized privilege gains and unauthorized access.

The security in Kubernetes relies heavily on correct configuration by administrators and developers and is not enabled by default. There have been real-world incidents that have demonstrated that insecure details, lax role-based access control(RABC) policies and unprotected control plane can allow attackers to compromise entire clusters. With this once the attackers gain access to the Kubernetes API or a privileged workload, they can deploy malicious containers, expand access through connected systems, or abuse cluster resources. 

Recent researches highlight the common occurrence of these security threats and the limitation of existing defense mechanism. \cite{Moric2025KubernetesSecurity} demonstrates that many Kubernetes compromise stems from the control plane misconfiguration and inadequate security hardening, reinforcing the need for organized security baseline and compliance centered configurations. \cite{Chen2025ShadowKube} shows that the traditional monitoring technique and rule-based defenses are often ineffective in Kubernetes environments, where they suggest the use of behavioral monitoring and honeypot to detect and stop attacks. \cite{Cesarano2025KubeFence} points out that the role-based access control in the Kubernetes API is to broad to protect, and recommends the use of API filtering to reduce the attacks. These studies show that there is not a single security system that will protect Kubernetes, but a combination of preventive hardening, reducing attacks surfaces and active defenses during runtime.

By building on these insight, this paper provides a systematic analysis of security hardening techniques and security strategies for containerized environments and Kubernetes based deployments.

This paper aims to:

\begin{itemize}
    \item Examine the key security vulnerabilities and potential attack surface for Kubernetes.
    \item Review and analyze recent approaches designed to address these issues.
    \item Compare these approaches, highlighting their strength, weaknesses, and aspects of deploying them.
    \item Present challenges and outline potential direction for future research to strengthen Kubernetes clusters.
\end{itemize}

\section{Background and Threat Model}

This section presents the technical background on containerized systems and Kubernetes, and defines threat model used throughout the paper.

\subsection{Virtualization and containerization}

Virtualization has long been used to improve resource utilization by runnung multiple virtual machines (VM) to run on a single host. The virtual machines each included their own guest operating system, which provided strong isolation between individual workloads, but at the cost of higher resource usage consumption and slower startup times. While it did improve the infrastructure flexibility, it does not support the scalability and rapid deployment that is required by modern cloud architecture.

Containerization addressed these limitations by enabling application to execute within isolated user spaces while sharing the host operating system kernel. Container engines such as Docker use Linux built isolation and resource limiting mechanism so that containers behave like in separate environment with the same operating system kernel. Compared to the virtual machines, they are faster, more lightweight, consume less resources. However, sharing the host kernel model also introduces new security risk, such as misconfigurations and vulnerabilities that can be exploited by attackers to allow them to escape the container isolation and access the host resources. \cite{Moric2025KubernetesSecurity}

\subsection{Container orchestration and Kubernetes}

As the container application becomes more widely adopted, the need to manage a large number of containers across distributed environements led to the development of a platform to manage these containers. Kubernetes emerged as the dominant orchestration system, automating deployment, management, recovery and scaling of containerized application.

Kubernetes operates using clusters that consist of a control plane and a set of worker nodes. The control plane consist of components such as API server, scheduler, control manager and the etcd key-value store that are critical to the management of the overall state of the clusters. Worker nodes run applications workloads in the form of pods, these pods group one or more containers together and share network and storage resources. Interactions with the clusters such as deploying application or changing the configuration file is done through the Kubernetes API.

Even though Kubernetes makes application management easier and more efficient through flexibility and automation, it also expands the systems attack surface. It exposes a wide variety of functionality that, if not properly secured, can be exploited by attackers. Studies \cite{Cesarano2025KubeFence} show that many vulnerabilities and misconfigurations are caused by API features, making the Kubernetes API a target for attackers. Furthermore, because Kubernetes workloads are constantly changing, making consistent policies is difficult, as shown in real-world deployment \cite{Chen2025ShadowKube}.

\subsection{Security Model of Kubernetes}

Kubernetes several built--in security mechanism specifically for the API, such as authentication, authorization and admission control. Authentication confirms the identify of the user and services interacting with the cluster, while authorization is generally handled by role-based access control (RBAC). RBAC decides permission for Kubernetes resources such as pods, deployment, and specifies which actions the users and services can perform.

However, RBAC only gives basic control of who can access resources, but does not limit how specific settings inside those resources can be used. This makes it possible for users with legitimate access to the resources to abuse advanced configurations to gain more privileges or exploit weaknesses. Security measures, such as network policies, pod security settings, and audit logging need to be manually set up and maintained by administrators. With these controls often applied inconsistently or turned off, it contributes significantly to Kubernetes security incidents \cite{Moric2025KubernetesSecurity}.


\section{Security Challenges in Containerized and Kubernetes Environment}

Despite it's wide adoption and powerful capabilities, Kubernetes introduces a number of security challenges which is a result of it's architectural design, configuration complexity and operational model.

\subsection{Container and Configuration Challenges}

One of the biggest security challenges with Kubernetes is the complexity of it's configuration models. Kubernetes depends on declarative configuration files, which are files where you tell the system what you want it to look like, rather than what to do. While this enables automation and scalability, it also makes mistakes more likely. To prevent security issues, administrators have to correctly configure parameters authentication, networking, storage and workload isolation.

In real-deployments, misconfigurations such as giving to many permisive RBAC roles, unsecured API endpoints, privileged containers, or setting up service accounts incorrectly are common. These misconfigurations can grant users or workloads excessive access, allowing attackers to gain control. Studies have demonstrated that many Kubernetes security challenges are not caused by unknown vulnerabilities, but by insecure default settings and configurations \cite{Moric2025KubernetesSecurity}.

\subsection{Kubernetes Control Plan and API Attack Surface}

For attackers the Kubernetes control plane is a high-value target due to it's central role in managing clusters and workload scheduling. The Kubernetes API manages all clusters and allows creating, modiyfing and deleting resources. If the API access control are not probably restricted, it may give access to attackers to abuse legitimate API calls to control workloads or manipulate cluster configuration.

Despite the fact that Kubernetes uses authentication and authorization mechanism, the API is still a critical attack surface. In some situations authenticated users with limited privileges can still misuse advanced configuration to cause it to behave unintended. A significant amount of vulnerabilities and misconfigurations are exercised through API specific features rather than the entire resources, this makes it difficult to control API access using the traditional access control methodd. Because the traditional access controls are broad it's hard to allow access and block others at a detailed level \cite{Cesarano2025KubeFence}.


\subsection{Limitations of Role-Based Access Control and Static Access Control}

The primary authorization mechanism used in Kubernetes to regulate access to cluster resources is role-based access control. The RBAC policies specify witch users or services that can perform actions on the Kubernetes resources such as pods, deployment or services. While the role-based access control provides a clear and standarized approach to access control, it has limitations in managing access in complex and rapidly changing Kubernetes environement.

A key limitation of RBAC is that it grants broad permission, without fine control over evert possible action. RBAC policies decides what kind of resources (pods, deployment or service) a user can access and what actions they can perform on them like (create, delete and modify). But it does not control the small details inside the resources such as the specific configuration option a user can choose. This means users with permission to create or modify resources can can misuse advanced configuration option to gain privileges or bypass isolation

RBAC is also a static access control, meaning it does not adjust to change in how workload behave or run. Since Kubernetes environement is dynamic, workloads are being created, scaled and removed constantly and because RBAC is a fixed, the policies cannot respond to unusual activity or misuse of granted permission at runtime. This limits it effectiveness against compromised workloads trusted users who misuse their small permissions to cause problems.

At a scale, managing these RBAC policies introduces operational challenges. To balance between security and usability when defining roles, administrators often grant people more access then they need to avoid slowing down development, which increases attack surface and weakens the clusters overall security. Recent studies \cite{Cesarano2025KubeFence} have highlighted the limitation with RBAC as a standalone defense mechanism, so additional approaches are need the provide finer control and while workloads are running.



\section{Security Hardening and Defense Strategies}

The security challenges described earlier have led to many strategies designed to improve security of containerized and Kubernetes environment. With recent studies demonstrating that relying on a single security mechanism is insufficient and requires the combination of preventive hardening, runtime and attack surface reduction. In this section we will review the security apporaches focused on the control plane and workload hardening, behavioral monitoring and detailed API access control.

\subsection{Control Plane and Workload Hardening}






\section{Comparative analysis}


\section{Future Research Direction}


\section{Conclusion}



\vspace{-12pt}

\bibliographystyle{ieeetr}
\bibliography{Ref}

\end{document}